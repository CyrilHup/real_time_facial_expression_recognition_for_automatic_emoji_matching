% chktex-file 8
% chktex-file 18
% chktex-file 26
\documentclass[12pt,a4paper]{article}
\usepackage[utf8]{inputenc}
\usepackage[T1]{fontenc}
\usepackage[french]{babel}
\usepackage{amsmath,amsfonts,amssymb}
\usepackage{graphicx}
\usepackage{listings}
\usepackage{xcolor}
\usepackage{hyperref}
\usepackage{geometry}
\usepackage{float}
\usepackage{booktabs}
\usepackage{caption}
\usepackage{subcaption}
\usepackage{algorithm}
\usepackage{algorithmic}
\usepackage{tikz}

\geometry{margin=2.5cm}

% Configuration des listings pour Python
\definecolor{codegreen}{rgb}{0,0.6,0}
\definecolor{codegray}{rgb}{0.5,0.5,0.5}
\definecolor{codepurple}{rgb}{0.58,0,0.82}
\definecolor{backcolour}{rgb}{0.95,0.95,0.92}

\lstdefinestyle{pythonstyle}{
    backgroundcolor=\color{backcolour},
    commentstyle=\color{codegreen},
    keywordstyle=\color{magenta},
    numberstyle=\tiny\color{codegray},
    stringstyle=\color{codepurple},
    basicstyle=\ttfamily\footnotesize,
    breakatwhitespace=false,
    breaklines=true,
    captionpos=b,
    keepspaces=true,
    numbers=left,
    numbersep=5pt,
    showspaces=false,
    showstringspaces=false,
    showtabs=false,
    tabsize=2,
    language=Python
}

\lstset{style=pythonstyle}

\title{\textbf{Reconnaissance d'Émotions Faciales en Temps Réel}\\
\large Rapport Technique - Deep Learning}
\author{Cyril}
\date{Novembre 2025}

\begin{document}

\maketitle

\begin{abstract}
Ce rapport présente le développement d'un système de reconnaissance d'émotions faciales en temps réel utilisant des réseaux de neurones convolutifs (CNN). Nous détaillons l'architecture initiale, les optimisations apportées, ainsi que les choix techniques effectués pour améliorer les performances de détection. Le système est capable de classifier huit émotions (colère, dégoût, peur, joie, tristesse, surprise, neutralité, et mépris) à partir d'un flux vidéo webcam, en utilisant une approche multi-sources unifiée (AffectNet + FER+).
\end{abstract}

\tableofcontents
\newpage

%==============================================================================
\section{Introduction}
%==============================================================================

La reconnaissance automatique des émotions faciales est un domaine en pleine expansion avec des applications dans l'interaction homme-machine, la santé mentale, le marketing, et l'éducation. Ce projet implémente un système complet de reconnaissance d'émotions en temps réel, depuis l'entraînement du modèle jusqu'à l'inférence via webcam.

\subsection{Objectifs}
\begin{itemize}
    \item Développer un modèle CNN capable de classifier 8 émotions faciales (7 de FER2013 + Contempt de FER+)
    \item Implémenter une application temps réel avec détection de visage
    \item Optimiser les performances pour une utilisation fluide
    \item Améliorer la précision par des techniques avancées de deep learning
\end{itemize}

%==============================================================================
\section{Architecture Initiale}
%==============================================================================

\subsection{Dataset : Approche Multi-Sources Unifiée}

Pour maximiser la robustesse du modèle, nous avons adopté une stratégie de fusion de datasets. Le système utilise désormais un **dataset unifié à 8 classes** combinant :

\begin{itemize}
    \item \textbf{Balanced AffectNet} : Source principale ($\sim$41k images, RGB 75$\times$75).
    \item \textbf{FER+} : Version améliorée de FER2013 avec labels corrigés par vote (Microsoft), upscalée à 75$\times$75.
\end{itemize}

\textbf{Avantages de cette approche :}
\begin{itemize}
    \item \textbf{Volume de données accru} : Combine la diversité d'AffectNet avec les cas difficiles de FER+.
    \item \textbf{Labels de haute qualité} : Utilisation exclusive de datasets vérifiés (FER+ vs FER2013 original).
    \item \textbf{Mapping Unifié} : Toutes les sources sont mappées vers les 8 émotions standards (Anger, Disgust, Fear, Happy, Sad, Surprise, Neutral, Contempt).
\end{itemize}

\subsection{Modèle Initial (\texttt{model.py})}

L'architecture initiale est un CNN simple avec 3 blocs convolutifs :

\begin{lstlisting}[caption={Architecture CNN pour Balanced AffectNet}]
class FaceEmotionCNN(nn.Module):
    def __init__(self, num_classes=8, in_channels=3, input_size=75):
        super(FaceEmotionCNN, self).__init__()
        # Bloc Convolutif 1 (75x75 -> 37x37)
        self.conv1a = nn.Conv2d(in_channels, 64, kernel_size=3, padding=1)
        self.bn1a = nn.BatchNorm2d(64)
        self.conv1b = nn.Conv2d(64, 64, kernel_size=3, padding=1)
        self.bn1b = nn.BatchNorm2d(64)
        self.pool1 = nn.MaxPool2d(2, 2)
        self.dropout1 = nn.Dropout(0.1)
        
        # ... Blocs 2, 3, 4 similaires ...
        
        # Global Average Pooling
        self.global_avg_pool = nn.AdaptiveAvgPool2d(1)
        
        # Couches Fully Connected
        self.fc1 = nn.Linear(512, 256)
        self.fc2 = nn.Linear(256, 128)
        self.fc3 = nn.Linear(128, num_classes)  # 8 emotions
\end{lstlisting}

\subsubsection{Analyse de l'architecture initiale}

\begin{table}[H]
\centering
\caption{Dimensions à travers le réseau (Balanced AffectNet)}
\begin{tabular}{lccc}
\toprule
\textbf{Couche} & \textbf{Entrée} & \textbf{Sortie} & \textbf{Paramètres} \\
\midrule
Block1 (2$\times$Conv) + Pool & $3 \times 75 \times 75$ & $64 \times 37 \times 37$ & 38,592 \\
Block2 (2$\times$Conv) + Pool & $64 \times 37 \times 37$ & $128 \times 18 \times 18$ & 147,712 \\
Block3 (2$\times$Conv) + Pool & $128 \times 18 \times 18$ & $256 \times 9 \times 9$ & 590,336 \\
Block4 (2$\times$Conv) + Pool & $256 \times 9 \times 9$ & $512 \times 4 \times 4$ & 2,360,320 \\
Global Avg Pool & $512 \times 4 \times 4$ & 512 & 0 \\
FC1 + BN & 512 & 256 & 131,584 \\
FC2 + BN & 256 & 128 & 33,024 \\
FC3 & 128 & 8 & 1,032 \\
\midrule
\textbf{Total} & & & \textbf{$\approx$ 3.3M} \\
\bottomrule
\end{tabular}
\end{table}

\textbf{Points positifs :}
\begin{itemize}
    \item Utilisation de Batch Normalization pour stabiliser l'entraînement
    \item Dropout (50\%) pour régulariser et éviter l'overfitting
    \item Architecture simple et rapide à entraîner
\end{itemize}

\textbf{Limitations :}
\begin{itemize}
    \item Peu de couches convolutives (faible capacité d'extraction de features)
    \item Nombre de filtres limité (32-64-128)
    \item Pas de régularisation dans les couches convolutives
    \item Couche FC1 très large (2.3M paramètres) créant un goulot d'étranglement
\end{itemize}

\subsection{Script d'entraînement initial (\texttt{train.py})}

\begin{lstlisting}[caption={Script d'entraînement initial}]
# Hyperparametres
BATCH_SIZE = 64
LEARNING_RATE = 0.001
EPOCHS = 25

# Transformations simples
transform = transforms.Compose([
    transforms.ToPILImage(),
    transforms.ToTensor(),
])

# Chargement du dataset
dataset = FER2013Dataset('./data/fer2013/fer2013.csv', transform=transform)
train_loader = DataLoader(dataset, batch_size=BATCH_SIZE, shuffle=True)

# Modele et optimiseur
model = FaceEmotionCNN().to(device)
criterion = nn.CrossEntropyLoss()
optimizer = optim.Adam(model.parameters(), lr=LEARNING_RATE)

# Boucle d'entrainement
for epoch in range(EPOCHS):
    running_loss = 0.0
    for inputs, labels in train_loader:
        inputs, labels = inputs.to(device), labels.to(device)
        optimizer.zero_grad()
        outputs = model(inputs)
        loss = criterion(outputs, labels)
        loss.backward()
        optimizer.step()
        running_loss += loss.item()
    
    print(f"Epoch {epoch+1}, Loss: {running_loss/len(train_loader)}")

torch.save(model.state_dict(), 'emotion_model.pth')
\end{lstlisting}

\textbf{Limitations de l'entraînement initial :}
\begin{itemize}
    \item Pas de data augmentation (risque d'overfitting)
    \item Pas de séparation train/validation (impossible de détecter l'overfitting)
    \item Learning rate fixe (convergence sous-optimale)
    \item Pas d'early stopping (gaspillage de ressources)
    \item Pas de gestion du déséquilibre des classes
\end{itemize}

\subsection{Application initiale (\texttt{app.py})}

L'application initiale effectue :
\begin{enumerate}
    \item Capture vidéo via webcam
    \item Détection de visage avec Haar Cascade
    \item Prétraitement de l'image (redimensionnement 48$\times$48, grayscale)
    \item Inférence avec le modèle CNN
    \item Affichage de l'émotion détectée avec emoji
\end{enumerate}

\begin{lstlisting}[caption={Boucle principale de l'application initiale}]
while True:
    ret, frame = cap.read()
    gray_frame = cv2.cvtColor(frame, cv2.COLOR_BGR2GRAY)
    faces = face_cascade.detectMultiScale(gray_frame, 
                                          scaleFactor=1.3, 
                                          minNeighbors=5)

    for (x, y, w, h) in faces:
        cv2.rectangle(frame, (x, y), (x + w, y + h), (255, 0, 0), 2)
        roi_gray = gray_frame[y:y+h, x:x+w]
        
        # Preprocessing
        roi_tensor = data_transform(roi_gray).unsqueeze(0).to(device)

        # Prediction
        with torch.no_grad():
            outputs = model(roi_tensor)
            probabilities = torch.nn.functional.softmax(outputs, dim=1)
            max_prob, predicted_idx = torch.max(probabilities, 1)
            
            idx = predicted_idx.item()
            confidence = max_prob.item() * 100
            emotion_text = emotion_dict[idx]
\end{lstlisting}

\textbf{Limitations :}
\begin{itemize}
    \item Prédictions instables (oscillations entre frames)
    \item Pas de normalisation de l'éclairage
    \item Paramètres de détection de visage sous-optimaux
    \item Interface utilisateur basique
\end{itemize}

%==============================================================================
\section{Optimisations de l'Architecture}
%==============================================================================

\subsection{Nouvelle Architecture CNN}

L'architecture améliorée comporte plusieurs optimisations majeures :

\begin{lstlisting}[caption={Architecture CNN améliorée}]
class FaceEmotionCNN(nn.Module):
    def __init__(self, num_classes=7):
        super(FaceEmotionCNN, self).__init__()
        
        # Bloc 1 (48x48 -> 24x24) - Double convolution
        self.conv1a = nn.Conv2d(1, 64, kernel_size=3, padding=1)
        self.bn1a = nn.BatchNorm2d(64)
        self.conv1b = nn.Conv2d(64, 64, kernel_size=3, padding=1)
        self.bn1b = nn.BatchNorm2d(64)
        self.pool1 = nn.MaxPool2d(2, 2)
        self.dropout1 = nn.Dropout(0.25)
        
        # Bloc 2 (24x24 -> 12x12)
        self.conv2a = nn.Conv2d(64, 128, kernel_size=3, padding=1)
        self.bn2a = nn.BatchNorm2d(128)
        self.conv2b = nn.Conv2d(128, 128, kernel_size=3, padding=1)
        self.bn2b = nn.BatchNorm2d(128)
        self.pool2 = nn.MaxPool2d(2, 2)
        self.dropout2 = nn.Dropout(0.25)
        
        # Bloc 3 (12x12 -> 6x6)
        self.conv3a = nn.Conv2d(128, 256, kernel_size=3, padding=1)
        self.bn3a = nn.BatchNorm2d(256)
        self.conv3b = nn.Conv2d(256, 256, kernel_size=3, padding=1)
        self.bn3b = nn.BatchNorm2d(256)
        self.pool3 = nn.MaxPool2d(2, 2)
        self.dropout3 = nn.Dropout(0.25)
        
        # Bloc 4 (6x6 -> 3x3)
        self.conv4a = nn.Conv2d(256, 512, kernel_size=3, padding=1)
        self.bn4a = nn.BatchNorm2d(512)
        self.conv4b = nn.Conv2d(512, 512, kernel_size=3, padding=1)
        self.bn4b = nn.BatchNorm2d(512)
        self.pool4 = nn.MaxPool2d(2, 2)
        self.dropout4 = nn.Dropout(0.25)
        
        # Global Average Pooling
        self.global_avg_pool = nn.AdaptiveAvgPool2d(1)
        
        # Fully Connected avec dropout progressif
        self.fc1 = nn.Linear(512, 256)
        self.bn_fc1 = nn.BatchNorm1d(256)
        self.dropout_fc1 = nn.Dropout(0.5)
        
        self.fc2 = nn.Linear(256, 128)
        self.bn_fc2 = nn.BatchNorm1d(128)
        self.dropout_fc2 = nn.Dropout(0.4)
        
        self.fc3 = nn.Linear(128, num_classes)
\end{lstlisting}

\subsubsection{Justification des choix architecturaux}

\begin{enumerate}
    \item \textbf{Double convolution par bloc (style VGG)} \\
    Inspiré de VGGNet, deux convolutions 3$\times$3 successives permettent un champ réceptif de 5$\times$5 avec moins de paramètres qu'une convolution 5$\times$5 unique :
    \begin{equation}
        \text{Paramètres }3\times3\times2 = 2 \times (3^2 \times C^2) = 18C^2
    \end{equation}
    \begin{equation}
        \text{Paramètres }5\times5 = 5^2 \times C^2 = 25C^2
    \end{equation}
    
    \item \textbf{Augmentation progressive des filtres (64→128→256→512)} \\
    Suit le principe que les premières couches extraient des features simples (bords, textures) tandis que les couches profondes capturent des concepts complexes (parties du visage, expressions).
    
    \item \textbf{Dropout progressif (0.25 dans conv, 0.5→0.4 dans FC)} \\
    Le dropout dans les couches convolutives (0.25) régularise sans trop perturber l'apprentissage spatial. Un dropout plus fort dans les FC (0.5) combat l'overfitting où le risque est le plus élevé.
    
    \item \textbf{Global Average Pooling (GAP)} \\
    Remplace le flatten traditionnel. Avantages :
    \begin{itemize}
        \item Réduit drastiquement les paramètres ($512 \times 3 \times 3 = 4608 \rightarrow 512$)
        \item Agit comme régularisateur
        \item Invariance spatiale accrue
    \end{itemize}
    
    \item \textbf{Initialisation Kaiming} \\
    Initialisation adaptée aux fonctions d'activation ReLU :
    \begin{equation}
        W \sim \mathcal{N}\left(0, \sqrt{\frac{2}{n_{in}}}\right)
    \end{equation}

    \item \textbf{Squeeze-and-Excitation (SE) Blocks} \\
    Intégration de modules d'attention SE qui recalibrent dynamiquement les features maps par canal, permettant au réseau de se concentrer sur les caractéristiques les plus pertinentes pour l'émotion.
\end{enumerate}

\begin{table}[H]
\centering
\caption{Comparaison des architectures}
\begin{tabular}{lcc}
\toprule
\textbf{Caractéristique} & \textbf{Initial} & \textbf{Amélioré} \\
\midrule
Blocs convolutifs & 3 & 4 \\
Convolutions par bloc & 1 & 2 \\
Filtres max & 128 & 512 \\
Paramètres totaux & $\approx$2.46M & $\approx$4.8M \\
Global Average Pooling & Non & Oui \\
Dropout conv & Non & Oui (0.25) \\
Batch Norm FC & Non & Oui \\
Attention & Non & SE-Blocks \\
\bottomrule
\end{tabular}
\end{table}

%==============================================================================
\section{Optimisations de l'Entraînement}
%==============================================================================

\subsection{Data Augmentation}

La data augmentation est cruciale pour un petit dataset comme FER2013. Elle augmente artificiellement la diversité des données d'entraînement.

\begin{lstlisting}[caption={Transformations de data augmentation}]
train_transform = transforms.Compose([
    transforms.ToPILImage(),
    transforms.RandomHorizontalFlip(p=0.5),
    transforms.RandomRotation(10),
    transforms.RandomAffine(
        degrees=0, 
        translate=(0.1, 0.1),
        scale=(0.9, 1.1)
    ),
    transforms.ColorJitter(brightness=0.2, contrast=0.2),
    transforms.ToTensor(),
])
\end{lstlisting}

\begin{table}[H]
\centering
\caption{Transformations de data augmentation}
\begin{tabular}{lp{8cm}}
\toprule
\textbf{Transformation} & \textbf{Justification} \\
\midrule
RandomHorizontalFlip (p=0.5) & Les expressions sont symétriques. Double effectivement le dataset. \\
RandomRotation (±10°) & Simule les légères inclinaisons de tête naturelles. \\
RandomAffine (translate) & Compense les variations de position du visage dans le cadre. \\
RandomAffine (scale 0.9-1.1) & Simule différentes distances caméra-visage. \\
ColorJitter (brightness, contrast) & Robustesse aux variations d'éclairage. \\
\bottomrule
\end{tabular}
\end{table}

\subsection{Séparation Train/Validation}

\begin{lstlisting}[caption={Séparation du dataset}]
VALIDATION_SPLIT = 0.15  # 15% pour validation

val_size = int(len(full_dataset) * VALIDATION_SPLIT)
train_size = len(full_dataset) - val_size

train_dataset, val_dataset = random_split(
    full_dataset, 
    [train_size, val_size],
    generator=torch.Generator().manual_seed(42)
)
\end{lstlisting}

\textbf{Importance :} La validation permet de :
\begin{itemize}
    \item Détecter l'overfitting (train loss $\downarrow$ mais val loss $\uparrow$)
    \item Sélectionner le meilleur modèle
    \item Ajuster les hyperparamètres
\end{itemize}

\subsection{Early Stopping}

L'early stopping arrête l'entraînement quand la validation ne s'améliore plus :

\begin{lstlisting}[caption={Implémentation de l'early stopping}]
PATIENCE = 7
best_val_acc = 0.0
patience_counter = 0

for epoch in range(EPOCHS):
    # ... entrainement ...
    val_loss, val_acc = validate(model, val_loader, criterion, device)
    
    if val_acc > best_val_acc:
        best_val_acc = val_acc
        patience_counter = 0
        torch.save(model.state_dict(), 'emotion_model_best.pth')
    else:
        patience_counter += 1
        if patience_counter >= PATIENCE:
            print("Early stopping triggered!")
            break
\end{lstlisting}

\subsection{Learning Rate Scheduler}

Le scheduler réduit le learning rate quand la validation stagne :

\begin{lstlisting}[caption={Learning rate scheduler}]
scheduler = optim.lr_scheduler.ReduceLROnPlateau(
    optimizer, 
    mode='min',      # Surveille la loss
    factor=0.5,      # Divise LR par 2
    patience=3       # Attend 3 epochs sans amelioration
)

# Dans la boucle d'entrainement
scheduler.step(val_loss)
\end{lstlisting}

\textbf{Principe :} Un LR élevé au début permet une convergence rapide, puis un LR plus faible permet un affinage précis.

\subsection{Optimiseur AdamW}

\begin{lstlisting}[caption={Optimiseur AdamW avec weight decay}]
optimizer = optim.AdamW(
    model.parameters(), 
    lr=LEARNING_RATE, 
    weight_decay=1e-4
)
\end{lstlisting}

AdamW corrige un problème de Adam : le weight decay est appliqué directement aux poids plutôt qu'au gradient, ce qui donne une meilleure régularisation L2.

\subsection{Gradient Clipping}

\begin{lstlisting}[caption={Gradient clipping pour la stabilité}]
loss.backward()
torch.nn.utils.clip_grad_norm_(model.parameters(), max_norm=1.0)
optimizer.step()
\end{lstlisting}

Empêche les gradients explosifs qui peuvent déstabiliser l'entraînement.

%==============================================================================
\section{Optimisations de l'Application Temps Réel}
%==============================================================================

\subsection{Lissage Temporel des Prédictions}

Le problème principal des prédictions frame-par-frame est l'instabilité : l'émotion affichée peut changer rapidement entre frames successives, créant un effet de \emph{flickering} désagréable.

\begin{lstlisting}[caption={Lissage temporel avec moyenne pondérée}]
from collections import deque

SMOOTHING_WINDOW = 5
prediction_history = deque(maxlen=SMOOTHING_WINDOW)

def get_smoothed_prediction(current_probs):
    prediction_history.append(current_probs.cpu().numpy())
    
    if len(prediction_history) < 2:
        return current_probs
    
    # Moyenne ponderee (frames recentes = plus de poids)
    weights = np.linspace(0.5, 1.0, len(prediction_history))
    weights = weights / weights.sum()
    
    smoothed = np.zeros(7)
    for i, probs in enumerate(prediction_history):
        smoothed += weights[i] * probs
    
    return torch.tensor(smoothed)
\end{lstlisting}

\textbf{Principe :} Au lieu d'utiliser uniquement la prédiction de la frame courante, on calcule une moyenne pondérée sur les $N$ dernières frames. Les frames récentes ont plus de poids pour maintenir la réactivité.

\subsection{Égalisation d'Histogramme}

\begin{lstlisting}[caption={Normalisation de l'éclairage}]
# Avant preprocessing
roi_gray = cv2.equalizeHist(roi_gray)
\end{lstlisting}

L'égalisation d'histogramme normalise la distribution des niveaux de gris, rendant le modèle plus robuste aux variations d'éclairage.

\begin{figure}[H]
\centering
\fbox{\parbox{0.8\textwidth}{\centering
\textbf{Image sombre} $\longrightarrow$ \textbf{Histogramme égalisé}
}}
\caption{L'égalisation redistribue les intensités sur toute la plage [0, 255]}
\end{figure}

\subsection{Paramètres de Détection Optimisés}

\begin{lstlisting}[caption={Paramètres de détection de visage optimisés}]
faces = face_cascade.detectMultiScale(
    gray_frame, 
    scaleFactor=1.1,   # Plus precis (etait 1.3)
    minNeighbors=5,
    minSize=(48, 48),  # Taille minimum
    flags=cv2.CASCADE_SCALE_IMAGE
)
\end{lstlisting}

\begin{table}[H]
\centering
\caption{Impact des paramètres de détection}
\begin{tabular}{lcc}
\toprule
\textbf{Paramètre} & \textbf{Valeur basse} & \textbf{Valeur haute} \\
\midrule
scaleFactor & Plus précis, plus lent & Moins précis, plus rapide \\
minNeighbors & Plus de faux positifs & Moins de détections \\
minSize & Détecte petits visages & Ignore petits visages \\
\bottomrule
\end{tabular}
\end{table}

\subsection{Interface Utilisateur Améliorée}

\begin{itemize}
    \item \textbf{Couleurs par émotion} : Chaque émotion a une couleur distinctive
    \item \textbf{Barres de progression} : Affichage des probabilités de toutes les classes
    \item \textbf{Fond pour le texte} : Améliore la lisibilité
\end{itemize}

\begin{lstlisting}[caption={Couleurs par émotion}]
emotion_colors = {
    0: (0, 0, 255),     # Rouge - Angry
    1: (0, 128, 0),     # Vert fonce - Disgust
    2: (128, 0, 128),   # Violet - Fear
    3: (0, 255, 255),   # Jaune - Happy
    4: (255, 0, 0),     # Bleu - Sad
    5: (0, 165, 255),   # Orange - Surprise
    6: (128, 128, 128)  # Gris - Neutral
}
\end{lstlisting}

%==============================================================================
\section{Amélioration du Dataset}
%==============================================================================

\subsection{Avantages de l'Approche Multi-Sources}

Par rapport à FER2013, notre dataset unifié présente plusieurs avantages majeurs :

\begin{enumerate}
    \item \textbf{Résolution supérieure} : Images 75$\times$75 pixels (vs 48$\times$48)
    \item \textbf{Images RGB} : 3 canaux de couleur offrant plus d'informations
    \begin{table}[H]
    \centering
    \begin{tabular}{lcc}
    \toprule
    \textbf{Émotion} & \textbf{Nombre d'images} & \textbf{Pourcentage} \\
    \midrule
    Angry & 5,126 & 12.5\% \\
    Contempt & 5,126 & 12.5\% \\
    Disgust & 5,126 & 12.5\% \\
    Fear & 5,126 & 12.5\% \\
    Happy & 5,126 & 12.5\% \\
    Neutral & 5,126 & 12.5\% \\
    Sad & 5,126 & 12.5\% \\
    Surprise & 5,126 & 12.5\% \\
    \bottomrule
    \end{tabular}
    \caption{Distribution des classes dans le dataset unifié.}
    \end{table}
    
    \item \textbf{Dataset équilibré} : Pas de biais de classe, contrairement à FER2013
    \item \textbf{8 classes natives} : Inclut Contempt (mépris) dès le départ
    \item \textbf{Meilleure qualité} : Annotations plus fiables
\end{enumerate}

\subsection{Chargement du Dataset}

Le dataset unifié est organisé en dossiers par émotion :

\begin{lstlisting}[caption={Utilisation du dataset unifié}]
class UnifiedDataset(Dataset):
    EMOTION_CLASSES = {
        'Anger': 0, 'Disgust': 1, 'Fear': 2, 'Happy': 3,
        'Sad': 4, 'Surprise': 5, 'Neutral': 6, 'Contempt': 7,
    }
    
    def __init__(self, root_dir, split='train', transform=None):
        self.images = []
        self.labels = []
        
        for emotion_name, emotion_idx in self.EMOTION_CLASSES.items():
            emotion_dir = os.path.join(root_dir, split, emotion_name)
            for img_name in os.listdir(emotion_dir):
                self.images.append(os.path.join(emotion_dir, img_name))
                self.labels.append(emotion_idx)
    
    def __getitem__(self, idx):
        image = Image.open(self.images[idx]).convert('RGB')
        image = np.array(image)  # 75x75x3
        label = self.labels[idx]
        
        if self.transform:
            image = self.transform(image=image)['image']
        
        return image, label
\end{lstlisting}

\subsection{Équilibrage des Classes}

Pour compenser le déséquilibre (notamment le manque de \textit{Disgust}), deux techniques sont implémentées :

\subsubsection{Poids de classe dans la loss}

\begin{lstlisting}[caption={CrossEntropyLoss avec poids de classe}]
def get_class_weights(dataset):
    class_counts = np.bincount(labels, minlength=7)
    weights = 1.0 / class_counts
    weights = weights / weights.sum() * len(weights)
    return torch.FloatTensor(weights)

class_weights = get_class_weights(train_dataset)
criterion = nn.CrossEntropyLoss(weight=class_weights)
\end{lstlisting}

Le poids de chaque classe est inversement proportionnel à sa fréquence :
\begin{equation}
    w_c = \frac{N}{N_c \times C}
\end{equation}
où $N$ est le nombre total d'échantillons, $N_c$ le nombre d'échantillons de la classe $c$, et $C$ le nombre de classes.

\subsubsection{WeightedRandomSampler}

\begin{lstlisting}[caption={Sampler équilibré}]
def get_balanced_sampler(dataset):
    class_counts = np.bincount(labels, minlength=7)
    weights = 1.0 / class_counts
    sample_weights = weights[labels]
    
    sampler = WeightedRandomSampler(
        weights=sample_weights,
        num_samples=len(sample_weights),
        replacement=True
    )
    return sampler
\end{lstlisting}

Le sampler sur-échantillonne les classes minoritaires pendant l'entraînement.

%==============================================================================
\section{Techniques d'Entraînement Avancées}
%==============================================================================

Pour améliorer significativement les performances, notamment sur les classes difficiles (Disgust, Contempt, Fear, Angry), nous avons développé un workflow d'entraînement avancé (notebook Jupyter) intégrant les techniques les plus récentes du deep learning.

\subsection{Focal Loss pour le Déséquilibre des Classes}

La Cross-Entropy standard traite tous les exemples de manière égale. La \textbf{Focal Loss} \cite{lin2017focal} ajoute un facteur de modulation qui réduit la contribution des exemples faciles et se concentre sur les exemples difficiles :

\begin{equation}
    \text{FL}(p_t) = -\alpha_t (1 - p_t)^\gamma \log(p_t)
\end{equation}

où $p_t$ est la probabilité prédite pour la vraie classe, $\alpha_t$ est le poids de la classe, et $\gamma$ (gamma) contrôle le focus sur les exemples difficiles.

\begin{lstlisting}[caption={Implémentation de la Focal Loss}]
class FocalLoss(nn.Module):
    def __init__(self, alpha=None, gamma=2.0, reduction='mean'):
        super().__init__()
        self.alpha = alpha  # Poids par classe
        self.gamma = gamma  # Focus parameter (2.0 recommande)
        self.reduction = reduction
    
    def forward(self, inputs, targets):
        ce_loss = F.cross_entropy(inputs, targets, 
                                  weight=self.alpha, 
                                  reduction='none')
        pt = torch.exp(-ce_loss)  # p_t
        focal_loss = ((1 - pt) ** self.gamma) * ce_loss
        
        if self.reduction == 'mean':
            return focal_loss.mean()
        return focal_loss
\end{lstlisting}

\textbf{Avantage :} Avec $\gamma = 2$, un exemple bien classifié ($p_t = 0.9$) contribue 100× moins qu'un exemple difficile ($p_t = 0.1$), permettant au modèle de se concentrer sur les émotions sous-représentées comme Disgust.

\subsection{Mixup : Régularisation par Interpolation}

Mixup \cite{zhang2018mixup} crée de nouveaux exemples d'entraînement en interpolant linéairement des paires d'images et leurs labels :

\begin{equation}
    \tilde{x} = \lambda x_i + (1 - \lambda) x_j
\end{equation}
\begin{equation}
    \tilde{y} = \lambda y_i + (1 - \lambda) y_j
\end{equation}

où $\lambda \sim \text{Beta}(\alpha, \alpha)$ avec $\alpha = 0.2$.

\begin{lstlisting}[caption={Implémentation de Mixup}]
def mixup_data(x, y, alpha=0.2):
    if alpha > 0:
        lam = np.random.beta(alpha, alpha)
    else:
        lam = 1

    batch_size = x.size(0)
    index = torch.randperm(batch_size).to(x.device)

    mixed_x = lam * x + (1 - lam) * x[index, :]
    y_a, y_b = y, y[index]
    return mixed_x, y_a, y_b, lam

def mixup_criterion(criterion, pred, y_a, y_b, lam):
    return lam * criterion(pred, y_a) + (1 - lam) * criterion(pred, y_b)
\end{lstlisting}

\subsection{CutMix : Augmentation par Découpage}

CutMix \cite{yun2019cutmix} est une variante de Mixup qui découpe et colle des régions rectangulaires entre images :

\begin{lstlisting}[caption={Implémentation de CutMix}]
def cutmix_data(x, y, alpha=1.0):
    lam = np.random.beta(alpha, alpha)
    batch_size = x.size(0)
    index = torch.randperm(batch_size).to(x.device)
    
    # Calcul de la boite de decoupe
    W, H = x.size(2), x.size(3)
    cut_rat = np.sqrt(1. - lam)
    cut_w = int(W * cut_rat)
    cut_h = int(H * cut_rat)
    
    cx = np.random.randint(W)
    cy = np.random.randint(H)
    
    bbx1 = np.clip(cx - cut_w // 2, 0, W)
    bby1 = np.clip(cy - cut_h // 2, 0, H)
    bbx2 = np.clip(cx + cut_w // 2, 0, W)
    bby2 = np.clip(cy + cut_h // 2, 0, H)
    
    x[:, :, bbx1:bbx2, bby1:bby2] = x[index, :, bbx1:bbx2, bby1:bby2]
    
    # Ajuster lambda selon la surface decoupee
    lam = 1 - ((bbx2-bbx1)*(bby2-bby1) / (W*H))
    return x, y, y[index], lam
\end{lstlisting}

\textbf{Avantage :} CutMix force le modèle à utiliser des indices partiels du visage (un œil, la bouche seule) plutôt que de dépendre de l'image entière, améliorant la robustesse.

\subsection{Label Smoothing}

Le label smoothing régularise en remplaçant les labels hard (one-hot) par des labels soft :

\begin{equation}
    y_\text{smooth} = (1 - \epsilon) \cdot y_\text{hard} + \frac{\epsilon}{K}
\end{equation}

avec $\epsilon = 0.1$ et $K = 8$ classes.

\begin{lstlisting}[caption={Application du Label Smoothing}]
class LabelSmoothingLoss(nn.Module):
    def __init__(self, classes, smoothing=0.1):
        super().__init__()
        self.smoothing = smoothing
        self.cls = classes
        
    def forward(self, pred, target):
        confidence = 1.0 - self.smoothing
        smooth_val = self.smoothing / (self.cls - 1)
        
        one_hot = torch.zeros_like(pred)
        one_hot.fill_(smooth_val)
        one_hot.scatter_(1, target.unsqueeze(1), confidence)
        
        log_prob = F.log_softmax(pred, dim=1)
        return -(one_hot * log_prob).sum(dim=1).mean()
\end{lstlisting}

\subsection{Augmentation Avancée avec Albumentations}

Nous utilisons la bibliothèque Albumentations pour des transformations plus sophistiquées :

\begin{lstlisting}[caption={Transformations Albumentations}]
import albumentations as A

train_transform = A.Compose([
    A.HorizontalFlip(p=0.5),
    A.ShiftScaleRotate(
        shift_limit=0.1,
        scale_limit=0.15,
        rotate_limit=15,
        p=0.5
    ),
    A.OneOf([
        A.MotionBlur(blur_limit=3, p=1.0),
        A.GaussianBlur(blur_limit=3, p=1.0),
        A.GaussNoise(var_limit=(10, 50), p=1.0),
    ], p=0.3),
    A.OneOf([
        A.RandomBrightnessContrast(
            brightness_limit=0.2,
            contrast_limit=0.2,
            p=1.0
        ),
        A.CLAHE(clip_limit=2.0, p=1.0),
    ], p=0.5),
    A.CoarseDropout(
        max_holes=1,
        max_height=12,
        max_width=12,
        fill_value=0,
        p=0.25
    ),
])
\end{lstlisting}

\begin{table}[H]
\centering
\caption{Nouvelles transformations Albumentations}
\begin{tabular}{lp{8cm}}
\toprule
\textbf{Transformation} & \textbf{Justification} \\
\midrule
MotionBlur / GaussianBlur & Simule le flou de mouvement en conditions réelles \\
GaussNoise & Robustesse au bruit de capteur \\
CLAHE & Améliore le contraste local, meilleur que l'égalisation standard \\
CoarseDropout & Similaire à Cutout, force la redondance des features \\
\bottomrule
\end{tabular}
\end{table}

\subsection{Gradient Accumulation}

Pour simuler des batch sizes plus grands sur GPU avec mémoire limitée :

\begin{lstlisting}[caption={Gradient Accumulation}]
ACCUMULATION_STEPS = 4  # Effective batch = 32 * 4 = 128

for i, (inputs, targets) in enumerate(train_loader):
    outputs = model(inputs)
    loss = criterion(outputs, targets)
    loss = loss / ACCUMULATION_STEPS
    loss.backward()
    
    if (i + 1) % ACCUMULATION_STEPS == 0:
        torch.nn.utils.clip_grad_norm_(
            model.parameters(), max_norm=1.0
        )
        optimizer.step()
        optimizer.zero_grad()
\end{lstlisting}

\textbf{Avantage :} Un batch effectif de 128 images stabilise les gradients et améliore la convergence.

\subsection{Cosine Annealing avec Warm Restarts}

Scheduler qui suit une courbe cosinus avec redémarrages périodiques :

\begin{lstlisting}[caption={CosineAnnealingWarmRestarts}]
scheduler = optim.lr_scheduler.CosineAnnealingWarmRestarts(
    optimizer,
    T_0=10,     # Premiere periode de 10 epochs
    T_mult=2,   # Periodes suivantes x2 (10, 20, 40...)
    eta_min=1e-6
)
\end{lstlisting}

\begin{figure}[H]
\centering
\fbox{\parbox{0.8\textwidth}{\centering
Learning Rate suit une courbe cosinus : \\
$\eta_t = \eta_\text{min} + \frac{1}{2}(\eta_\text{max} - \eta_\text{min})(1 + \cos(\frac{T_\text{cur}}{T_i}\pi))$
}}
\caption{Le learning rate diminue selon une courbe cosinus puis remonte à chaque restart}
\end{figure}

\subsection{Optimisations Matérielles et Vitesse}

Pour accélérer l'entraînement sur des datasets volumineux, nous avons exploité les capacités modernes de PyTorch 2.0+ :

\begin{itemize}
    \item \textbf{Automatic Mixed Precision (AMP)} : Utilisation de \texttt{float16} pour les calculs tensoriels, réduisant l'empreinte mémoire et accélérant les opérations sur GPU (Tensor Cores).
    \item \textbf{torch.compile} : Compilation JIT du modèle avec le mode \texttt{max-autotune}, optimisant le graphe de calcul pour l'architecture spécifique du GPU.
    \item \textbf{Optimisations CUDA} : Activation de TF32 (TensorFloat-32) et tuning des kernels cuDNN.
\end{itemize}

Ces optimisations permettent d'utiliser des batch sizes plus importants (jusqu'à 1536) et de réduire le temps d'entraînement par époque.

%==============================================================================
\section{Application Étendue avec MediaPipe}
%==============================================================================

Pour améliorer la détection des émotions difficiles et enrichir le panel d'emojis affichables, nous avons intégré \textbf{MediaPipe}, la bibliothèque de Google pour l'analyse faciale et gestuelle en temps réel.

\subsection{MediaPipe Face Mesh : Analyse des Landmarks Faciaux}

Face Mesh détecte 468 points de repère 3D sur le visage, permettant une analyse fine des caractéristiques faciales :

\begin{lstlisting}[caption={Initialisation de Face Mesh}]
import mediapipe as mp

mp_face_mesh = mp.solutions.face_mesh
face_mesh = mp_face_mesh.FaceMesh(
    max_num_faces=1,
    refine_landmarks=True,  # Points supplementaires iris
    min_detection_confidence=0.5,
    min_tracking_confidence=0.5
)
\end{lstlisting}

\subsubsection{Features Extraites}

\begin{lstlisting}[caption={Classe FacialAnalyzer pour l'extraction de features}]
class FacialAnalyzer:
    # Indices des landmarks cles
    LEFT_EYE = [33, 160, 158, 133, 153, 144]
    RIGHT_EYE = [362, 385, 387, 263, 373, 380]
    MOUTH = [61, 291, 0, 17, 78, 308]
    LEFT_EYEBROW = [70, 63, 105, 66, 107]
    RIGHT_EYEBROW = [336, 296, 334, 293, 300]
    
    def analyze(self, landmarks):
        features = {}
        
        # Ouverture des yeux (Eye Aspect Ratio)
        features['left_eye_open'] = self._eye_aspect_ratio(
            [landmarks[i] for i in self.LEFT_EYE]
        )
        features['right_eye_open'] = self._eye_aspect_ratio(
            [landmarks[i] for i in self.RIGHT_EYE]
        )
        
        # Ouverture de la bouche
        features['mouth_open'] = self._mouth_aspect_ratio(landmarks)
        
        # Position des sourcils
        features['brow_raise'] = self._brow_raise(landmarks)
        features['brow_squeeze'] = self._brow_squeeze(landmarks)
        
        # Sourire (ratio largeur/hauteur bouche)
        features['smile'] = self._smile_ratio(landmarks)
        
        return features
    
    def _eye_aspect_ratio(self, eye_points):
        # Ratio vertical/horizontal pour detecter clignement
        vertical = np.linalg.norm(
            np.array(eye_points[1]) - np.array(eye_points[5])
        )
        horizontal = np.linalg.norm(
            np.array(eye_points[0]) - np.array(eye_points[3])
        )
        return vertical / (horizontal + 1e-6)
\end{lstlisting}

\subsubsection{Boost des Émotions Difficiles}

Les features faciales sont utilisées pour renforcer la détection des émotions que le CNN a du mal à identifier :

\begin{lstlisting}[caption={Système de boost des émotions}]
EMOTION_BOOST = {
    'Angry': 1.4,     # Boost de 40%
    'Disgust': 1.5,   # Boost de 50%
    'Fear': 1.3,      # Boost de 30%
    'Contempt': 1.4,  # Boost de 40%
    'Neutral': 0.85   # Reduction de 15%
}

def boost_emotions(probs, features):
    boosted = probs.copy()
    
    # Boost Angry si sourcils fronces
    if features.get('brow_squeeze', 0) > 0.6:
        boosted[ANGRY_IDX] *= 1.3
    
    # Boost Disgust si nez fronce
    if features.get('nose_wrinkle', 0) > 0.5:
        boosted[DISGUST_IDX] *= 1.4
    
    # Boost Fear si yeux grands ouverts
    avg_eye = (features['left_eye_open'] + 
               features['right_eye_open']) / 2
    if avg_eye > 0.4:
        boosted[FEAR_IDX] *= 1.2
    
    return boosted / boosted.sum()  # Re-normaliser
\end{lstlisting}

\subsection{Extension du Mapping Emoji}

Au-delà des 8 émotions de base, les features faciales permettent d'afficher des emojis plus nuancés :

\begin{lstlisting}[caption={Mapping emoji étendu}]
EXTENDED_EMOJI_MAP = {
    # Emotions de base
    'Happy': '😊',
    'Sad': '😢',
    'Angry': '😠',
    'Surprise': '😲',
    'Fear': '😨',
    'Disgust': '🤢',
    'Neutral': '😐',
    'Contempt': '😏',
    
    # Combinaisons basees sur features
    'very_happy': '😁',      # Sourire large
    'laughing': '😂',        # Bouche ouverte + sourire
    'thinking': '🤔',        # Un sourcil leve
    'sleepy': '😴',          # Yeux fermes
    'wink': '😉',            # Un oeil ferme
    'kiss': '😘',            # Bouche en O
    'love': '😍',            # Coeurs si smile fort
    'cool': '😎',            # Neutre + confiant
    'skeptical': '🤨',       # Sourcil asymetrique
}

def get_extended_emoji(emotion, features):
    # Verifier conditions speciales
    if features['left_eye_open'] < 0.15:
        if features['right_eye_open'] > 0.25:
            return EXTENDED_EMOJI_MAP['wink']
    
    if features['smile'] > 0.7 and features['mouth_open'] > 0.4:
        return EXTENDED_EMOJI_MAP['laughing']
    
    if emotion == 'Happy' and features['smile'] > 0.8:
        return EXTENDED_EMOJI_MAP['very_happy']
    
    return EXTENDED_EMOJI_MAP.get(emotion, '❓')
\end{lstlisting}

\subsection{MediaPipe Hands : Détection des Gestes}

Pour enrichir l'interaction, nous avons ajouté la détection des gestes de la main :

\begin{lstlisting}[caption={Initialisation de MediaPipe Hands}]
mp_hands = mp.solutions.hands
hands = mp_hands.Hands(
    static_image_mode=False,
    max_num_hands=2,
    min_detection_confidence=0.7,
    min_tracking_confidence=0.5
)
\end{lstlisting}

\subsubsection{Gestes Reconnus}

\begin{lstlisting}[caption={Classe HandRecognizer}]
class HandRecognizer:
    def recognize_gesture(self, landmarks):
        # Extraire les etats des doigts (leve/baisse)
        fingers = self._get_finger_states(landmarks)
        thumb, index, middle, ring, pinky = fingers
        
        # Pouce leve
        if thumb and not any([index, middle, ring, pinky]):
            return 'thumbs_up', '👍'
        
        # Signe de paix (V)
        if index and middle and not ring and not pinky:
            return 'peace', '✌️'
        
        # Signe OK
        if self._is_ok_gesture(landmarks):
            return 'ok', '👌'
        
        # Rock (cornes)
        if index and pinky and not middle and not ring:
            return 'rock', '🤘'
        
        # Shaka (pouce + auriculaire)
        if thumb and pinky and not index and not middle:
            return 'shaka', '🤙'
        
        # Poing ferme
        if not any(fingers):
            return 'fist', '✊'
        
        # Main ouverte (tous les doigts)
        if all(fingers):
            return 'wave', '👋'
        
        # Pointer
        if index and not middle and not ring and not pinky:
            return 'point', '👆'
        
        return None, None
    
    def _get_finger_states(self, landmarks):
        # Comparer position des bouts vs articulations
        thumb = landmarks[4].y < landmarks[3].y
        index = landmarks[8].y < landmarks[6].y
        middle = landmarks[12].y < landmarks[10].y
        ring = landmarks[16].y < landmarks[14].y
        pinky = landmarks[20].y < landmarks[18].y
        return [thumb, index, middle, ring, pinky]
\end{lstlisting}

\begin{table}[H]
\centering
\caption{Gestes reconnus et emojis correspondants}
\begin{tabular}{lcc}
\toprule
\textbf{Geste} & \textbf{Description} & \textbf{Emoji} \\
\midrule
Thumbs Up & Pouce levé seul & 👍 \\
Peace & Index et majeur levés & ✌️ \\
OK & Pouce et index formant un cercle & 👌 \\
Rock & Index et auriculaire levés & 🤘 \\
Shaka & Pouce et auriculaire & 🤙 \\
Wave & Tous les doigts levés & 👋 \\
Fist & Poing fermé & ✊ \\
Point & Index seul levé & 👆 \\
\bottomrule
\end{tabular}
\end{table}

%==============================================================================
\section{Optimisations de l'Inférence (\texttt{app\_v3.py})}
%==============================================================================

La version 3 de l'application intègre plusieurs optimisations pour améliorer la précision et la stabilité des prédictions.

\subsection{Test-Time Augmentation (TTA)}

Le TTA applique plusieurs transformations à l'image d'entrée et moyenne les prédictions :

\begin{lstlisting}[caption={Implémentation du TTA}]
class EmotionClassifier:
    def predict_with_tta(self, face_roi, num_augmentations=3):
        predictions = []
        
        # Prediction originale
        predictions.append(self._predict_single(face_roi))
        
        # Flip horizontal
        flipped = cv2.flip(face_roi, 1)
        predictions.append(self._predict_single(flipped))
        
        # Legeres variations de luminosite
        for _ in range(num_augmentations - 2):
            factor = np.random.uniform(0.9, 1.1)
            adjusted = np.clip(face_roi * factor, 0, 255)
            predictions.append(
                self._predict_single(adjusted.astype(np.uint8))
            )
        
        # Moyenne des predictions
        return np.mean(predictions, axis=0)
\end{lstlisting}

\textbf{Avantage :} Le TTA réduit la variance des prédictions et améliore la robustesse aux petites perturbations.

\subsection{Prétraitement CLAHE}

Contrairement à l'égalisation d'histogramme globale, CLAHE (Contrast Limited Adaptive Histogram Equalization) travaille sur des régions locales :

\begin{lstlisting}[caption={Application de CLAHE}]
def preprocess_face(self, face_roi):
    # Redimensionner
    face = cv2.resize(face_roi, (48, 48))
    
    # CLAHE pour le contraste local
    clahe = cv2.createCLAHE(clipLimit=2.0, tileGridSize=(4, 4))
    face = clahe.apply(face)
    
    # Normalisation
    face = face.astype(np.float32) / 255.0
    
    return face
\end{lstlisting}

\begin{table}[H]
\centering
\caption{Comparaison des méthodes d'égalisation}
\begin{tabular}{lp{5cm}p{5cm}}
\toprule
\textbf{Méthode} & \textbf{Avantages} & \textbf{Inconvénients} \\
\midrule
Histogramme global & Simple, rapide & Peut sur-amplifier le bruit \\
CLAHE & Préserve les détails locaux, contrôle le contraste & Légèrement plus lent \\
\bottomrule
\end{tabular}
\end{table}

\subsection{Lissage Temporel Amélioré}

La version 3 utilise un lissage exponentiel plus sophistiqué :

\begin{lstlisting}[caption={Lissage temporel avec pondération exponentielle}]
SMOOTHING_WINDOW = 7

def get_smoothed_prediction(self, current_probs):
    self.prediction_history.append(current_probs)
    
    if len(self.prediction_history) < 2:
        return current_probs
    
    # Poids exponentiels (frames recentes plus importantes)
    n = len(self.prediction_history)
    weights = np.exp(np.linspace(-1, 0, n))
    weights = weights / weights.sum()
    
    smoothed = np.zeros(8)
    for i, probs in enumerate(self.prediction_history):
        smoothed += weights[i] * probs
    
    return smoothed
\end{lstlisting}

\subsection{Architecture Modulaire}

L'application V3 sépare clairement les responsabilités :

\begin{lstlisting}[caption={Architecture modulaire de app\_v3.py}]
class EmotionClassifier:
    """Gestion du modele CNN et predictions"""
    pass

class FacialAnalyzer:
    """Analyse des landmarks faciaux avec MediaPipe"""
    pass

class HandRecognizer:
    """Detection des gestes de la main"""
    pass

class EmotionApp:
    """Application principale orchestrant les composants"""
    def __init__(self):
        self.classifier = EmotionClassifier()
        self.facial_analyzer = FacialAnalyzer()
        self.hand_recognizer = HandRecognizer()
    
    def process_frame(self, frame):
        # 1. Detecter visage
        # 2. Analyser features faciales
        # 3. Predire emotion avec CNN + TTA
        # 4. Boost avec features
        # 5. Detecter gestes
        # 6. Afficher resultats
        pass
\end{lstlisting}

\subsection{Datasets Alternatifs}

Pour des performances encore meilleures, d'autres datasets sont supportés :

\begin{table}[H]
\centering
\caption{Comparaison des datasets d'émotions faciales}
\begin{tabular}{lcccl}
\toprule
\textbf{Dataset} & \textbf{Images} & \textbf{Résolution} & \textbf{Classes} & \textbf{Accès} \\
\midrule
FER2013 & 35,887 & 48$\times$48 (Gray) & 7 & Gratuit (Kaggle) \\
FER+ & 35,887 & 48$\times$48 (Gray) & 8 & Gratuit (GitHub) \\
\textbf{Dataset Unifié} & \textbf{41,008} & \textbf{75$\times$75 (RGB)} & \textbf{8} & \textbf{Gratuit (Kaggle)} \\
AffectNet (original) & 450,000 & Variable & 8 & Demande requise \\
RAF-DB & 30,000 & 100$\times$100 & 7 & Demande requise \\
\bottomrule
\end{tabular}
\end{table}

\begin{lstlisting}[caption={Support multi-datasets}]
class AffectNetDataset(Dataset):
    # Mapping AffectNet vers les 7 classes FER
    AFFECTNET_TO_FER = {
        0: 6,  # Neutral
        1: 3,  # Happy
        2: 4,  # Sad
        3: 5,  # Surprise
        4: 2,  # Fear
        5: 1,  # Disgust
        6: 0,  # Anger
        7: 6,  # Contempt -> Neutral
    }
    
    def __getitem__(self, idx):
        image = Image.open(self.images[idx]).convert('L')
        image = image.resize((48, 48))
        label = self.AFFECTNET_TO_FER[self.labels[idx]]
        return image, label
\end{lstlisting}

%==============================================================================
\section{Résultats Attendus}
%==============================================================================

\subsection{Amélioration de la Précision}

\begin{table}[H]
\centering
\caption{Amélioration attendue de la précision avec le Dataset Unifié}
\begin{tabular}{lcc}
\toprule
\textbf{Configuration} & \textbf{Précision estimée} & \textbf{Gain} \\
\midrule
FER2013 baseline (grayscale 48$\times$48) & 60-65\% & - \\
Dataset Unifié (RGB 75$\times$75) & 70-75\% & +10\% \\
+ Data augmentation RGB & 73-78\% & +3\% \\
+ Mixup/CutMix & 76-80\% & +3\% \\
+ Label Smoothing & 78-82\% & +2\% \\
+ TTA à l'inférence & 80-84\% & +2\% \\
\midrule
\textbf{Total avec optimisations} & \textbf{80-84\%} & \textbf{+20-24\%} \\
\bottomrule
\end{tabular}
\end{table}

\subsection{Amélioration des Classes Difficiles}

\begin{table}[H]
\centering
\caption{Impact attendu sur les classes (avec dataset équilibré)}
\begin{tabular}{lccc}
\toprule
\textbf{Émotion} & \textbf{FER2013 Baseline} & \textbf{Dataset Unifié} & \textbf{Gain} \\
\midrule
Angry & 55\% & 75\% & +20\% \\
Disgust & 35\% & 72\% & +37\% \\
Fear & 45\% & 70\% & +25\% \\
Contempt & 40\% & 68\% & +28\% \\
\bottomrule
\end{tabular}
\end{table}

\textit{Note : L'amélioration majeure sur Disgust et Contempt est due au dataset parfaitement équilibré et aux images RGB de meilleure qualité.}

\subsection{Amélioration de l'Expérience Utilisateur}

\begin{itemize}
    \item \textbf{Stabilité} : Le lissage temporel amélioré élimine le \emph{flickering}
    \item \textbf{Robustesse} : CLAHE gère mieux les variations d'éclairage que l'égalisation standard
    \item \textbf{Précision TTA} : Le test-time augmentation réduit la variance des prédictions
    \item \textbf{Emojis étendus} : Plus de 15 emojis possibles grâce à l'analyse des features faciales
    \item \textbf{Gestes de la main} : 8 gestes reconnus avec emojis correspondants
    \item \textbf{Lisibilité} : Les couleurs et barres de progression améliorent la compréhension
    \item \textbf{Réactivité} : Les paramètres de détection optimisés améliorent la fluidité
\end{itemize}

%==============================================================================
\section{Conclusion}
%==============================================================================

Ce projet a permis de développer un système complet et avancé de reconnaissance d'émotions faciales en temps réel. Les principales contributions sont :

\begin{enumerate}
    \item \textbf{Architecture CNN optimisée} avec double convolution, dropout progressif, et Global Average Pooling
    
    \item \textbf{Pipeline d'entraînement avancé} incluant :
    \begin{itemize}
        \item Focal Loss pour le déséquilibre des classes
        \item Mixup et CutMix pour la régularisation
        \item Label Smoothing pour la généralisation
        \item Augmentations avancées avec Albumentations
    \end{itemize}
    
    \item \textbf{Intégration MediaPipe} pour :
    \begin{itemize}
        \item Analyse des 468 landmarks faciaux (Face Mesh)
        \item Détection de 8 gestes de la main
        \item Boost contextuel des émotions difficiles
    \end{itemize}
    
    \item \textbf{Application temps réel optimisée (V3)} avec :
    \begin{itemize}
        \item Test-Time Augmentation (TTA)
        \item Prétraitement CLAHE
        \item Lissage temporel exponentiel
        \item Mapping emoji étendu (15+ emojis)
    \end{itemize}
    
    \item \textbf{Support multi-datasets} permettant d'utiliser FER+, AffectNet, ou RAF-DB
\end{enumerate}

\subsection{Perspectives}

Pour aller plus loin, les améliorations suivantes pourraient être envisagées :

\begin{itemize}
    \item Utilisation de transfer learning (VGGFace, ResNet pré-entraîné sur visages)
    \item Architecture avec attention mechanism (Transformer, CBAM)
    \item Détection multi-tâches (émotion + âge + genre)
    \item Analyse de la valence et de l'arousal (modèle dimensionnel)
    \item Déploiement sur edge devices (quantification INT8, pruning)
    \item Intégration d'un modèle audio pour l'analyse multimodale
\end{itemize}

%==============================================================================
\appendix
\section{Structure du Projet}
%==============================================================================

\begin{verbatim}
Final_project/
|-- app.py                   # Application temps reel (version initiale)
|-- app_extended.py          # Version avec MediaPipe Face Mesh
|-- app_extended_v2.py       # Version avec gestes de la main
|-- app_v3.py                # Version optimisee (TTA, CLAHE, modulaire)
|-- model.py                 # Architecture CNN (RGB 75x75)
|-- train_affectnet_notebook.ipynb # Notebook d'entrainement (Multi-source)
|-- train_affectnet.py       # Script d'entrainement (Legacy)
|-- train.py                 # Script d'entrainement (Legacy FER2013)
|-- dataset_affectnet.py     # Dataset Balanced AffectNet
|-- dataset.py               # Dataset FER2013 (legacy)
|-- download_datasets.py     # Script de telechargement des datasets
|-- emotion_model.pth        # Poids du modele entraine
|-- emotion_model_best.pth   # Meilleur modele (early stopping)
|-- data/
|   +-- affectnet/           # Dataset Balanced AffectNet
|       |-- train/
|       |   |-- Anger/
|       |   |-- Contempt/
|       |   |-- Disgust/
|       |   |-- Fear/
|       |   |-- Happy/
|       |   |-- Neutral/
|       |   |-- Sad/
|       |   +-- Surprise/
|       |-- val/
|       +-- test/
|-- report/
|   +-- report.tex           # Ce rapport
+-- README.md
\end{verbatim}

%==============================================================================
\section{Références}
%==============================================================================

\begin{enumerate}
    \item Goodfellow, I. J., et al. (2013). "Challenges in representation learning: A report on three machine learning contests." \textit{ICML Workshop}.
    
    \item Barsoum, E., et al. (2016). "Training Deep Networks for Facial Expression Recognition with Crowd-Sourced Label Distribution." \textit{ACM ICMI}.
    
    \item Mollahosseini, A., et al. (2017). "AffectNet: A Database for Facial Expression, Valence, and Arousal Computing in the Wild." \textit{IEEE Trans. Affective Computing}.
    
    \item Li, S., et al. (2017). "Reliable Crowdsourcing and Deep Locality-Preserving Learning for Expression Recognition in the Wild." \textit{CVPR}.
    
    \item He, K., et al. (2015). "Delving Deep into Rectifiers: Surpassing Human-Level Performance on ImageNet Classification." \textit{ICCV}.
    
    \item Lin, T. Y., et al. (2017). "Focal Loss for Dense Object Detection." \textit{ICCV}. \label{lin2017focal}
    
    \item Zhang, H., et al. (2018). "mixup: Beyond Empirical Risk Minimization." \textit{ICLR}. \label{zhang2018mixup}
    
    \item Yun, S., et al. (2019). "CutMix: Regularization Strategy to Train Strong Classifiers with Localizable Features." \textit{ICCV}. \label{yun2019cutmix}
    
    \item Lugaresi, C., et al. (2019). "MediaPipe: A Framework for Building Perception Pipelines." \textit{arXiv preprint arXiv:1906.08172}.
    
    \item Buslaev, A., et al. (2020). "Albumentations: Fast and Flexible Image Augmentations." \textit{Information}, 11(2), 125.
\end{enumerate}

\end{document}
